\documentclass[a4paper,twoside,phd]{BYUPhys}
% The BYUPhys class is for producing theses and dissertations
% in the BYU Department of Physics and Astronomy.  You can supply
% the following optional arguments in the square brackets to
% specify the thesis type:
%
%   senior  : Produces the senior thesis preliminary pages (default)
%   honors  : Produces the honors thesis preliminary pages
%   masters : Produces the masters thesis preliminary pages
%   phd     : Produces the PhD dissertation preliminary pages
%
% The default format is appropriate for printing, with blank pages
% inserted after the preliminary pages in twoside mode so you can
% send it directly to a two-sided printer. However, for ETD
% submission the blank pages need to be removed from the final output.
% The following option does this for you:
%
%   etd     : Produces a copy with no blank pages in the preliminary section.
%             Remove this option to produce a version with blank pages inserted
%             for easy double sided printing.
%
% The rest of the class options are the same as the regular book class.
% A few to remember:
%
%   oneside : Produces single sided print layout (recommended for theses less than 50 pages)
%   twoside : Produces double sided print layout (the default if you remove oneside)
%
% The BYUPhys class provides the following macros:
%
%   \makepreliminarypages : Makes the preliminary pages
%   \clearemptydoublepage : same as \cleardoublepage but doesn't put page numbers
%                           on blank intervening pages
%   \singlespace          : switch to single spaced lines
%   \doublespace          : switch to double spaced lines
%
% --------------------------- Load Packages ---------------------------------

% The graphicx package allows the inclusion of figures.  Plain LaTeX and
% pdfLaTeX handle graphics differently. The following code checks which one
% you are compiling with, and switches the graphicx package options accordingly.
\usepackage{ifpdf}
\ifpdf
  \usepackage[pdftex]{graphicx}
\else
  \usepackage[dvips]{graphicx}
\fi

%%%%%%%%%%%%%%%%%%%%%%%%%%%%%%%%%%%%%%%%%%%%%%%%%%%%%%%%%%%%%%%%%%
% Edited : Beeshanga
%
% If you need to include any code in the text use this package
% \usepackage{listings}
% It can be used to make key words bold, add colours, etc. Refer
% to http://en.wikibooks.org/wiki/LaTeX/Packages/Listings for
% more information.
%
% For theorems, propositions, proofs and assumtions use this
% package
% \usepackage{amsthm}
% For more information refer to the following website
% http://en.wikibooks.org/wiki/LaTeX/Theorems
%
%%%%%%%%%%%%%%%%%%%%%%%%%%%%%%%%%%%%%%%%%%%%%%%%%%%%%%%%%%%%%%%%%%

% The fancyhdr package allows you to easily customize the page header.
% The settings below produce a nice, well separated header.
\usepackage{fancyhdr}
  \fancyhead{}
  \fancyhead[LO]{\slshape \rightmark}
  \fancyhead[RO,LE]{\textbf{\thepage}}
  \fancyhead[RE]{\slshape \leftmark}
  \fancyfoot{}
  \pagestyle{fancy}
  \renewcommand{\chaptermark}[1]{\markboth{\chaptername \ \thechapter. #1}{}}
  \renewcommand{\sectionmark}[1]{\markright{\thesection \ #1}}


% The cite package cleans up the way citations are handled.  For example, it
% changes the citation [1,2,3,6,7,8,9,10,11] into [1-3,6-11].  If your advisor
% wants superscript citations, use the overcite package instead of the cite package.
\usepackage{cite}

% The makeidx package makes your index for you.  To make an index entry,
% go to the place in the book that should be referenced and type
%  \index{key}
% An index entry labeled "key" (or whatever you type) will then
% be included and point to the correct page.
%\usepackage{makeidx}
%\makeindex

% The url package allows for the nice typesetting of URLs.  Since URLs are often
% long with no spaces, they mess up line wrapping.  The command \url{http://www.physics.byu.edu}
% allows LaTeX to break the url across lines at appropriate places: e.g. http://www.
% physics.byu.edu.  This is helpful if you reference web pages.
\usepackage{url}
\urlstyle{rm}

% If you have a lot of equations, you might be interested in the amstex package.
% It defines a number of environments and macros that are helpful for mathematics.
% We don't do much math in this example, so we haven't used amstex here.
\usepackage{amsmath}
\usepackage{amssymb}
\usepackage{subfigure}
\usepackage{cite}
\usepackage{amsxtra}
\usepackage{amsfonts}
\usepackage{graphicx}
\usepackage{multirow} % This is package for multi-rows in tables added on 7th July 2009 by Arif
%\usepackage{setspace}

% The caption package allows us to change the formatting of figure captions.
% The commands here change to the suggested caption format: single spaced and a bold tag
\usepackage[labelfont=bf,labelsep=colon]{caption}%[2008/04/01]
 \DeclareCaptionFormat{suggested}{\singlespace#1#2#3\par\doublespace}
 \captionsetup{format=suggested}


\usepackage{array}
\usepackage{multirow}
\usepackage{verbatim}
\usepackage{enumerate}

% Defining the symbols

% The hyperref package provides automatic linking and bookmarking for the table
% of contents, index, equation references, and figure references.  It must be
% included for the BYU Physics class to make a properly functioning electronic
% thesis.  It should be the last package loaded if possible.
%
% To include a link in your pdf use \href{URL}{Text to be displayed}.  If your
% display text is the URL, you probably should use the \url{} command discussed
% above.
%
% To add a bookmark in the pdf you can use \pdfbookmark.  You can look up its usage
% in the hyperref package documentation
\usepackage[bookmarksnumbered,pdfpagelabels=true,plainpages=false,colorlinks=true,
            linkcolor=black,citecolor=red,urlcolor=blue]{hyperref}

% ------------------------- Fill in these fields for the preliminary pages ----------------------------
%
% For Senior and honors this is the year and month that you submit the thesis
% For Masters and PhD, this is your graduation date
  \Year{2018}
  \Month{November XX,}
  \Author{Dione Morales}

% If you have a long title, split it between two lines. The \TitleBottom field defines the second line
% A two line title should be an "inverted pyramid" with the top line longer than the bottom.
  \TitleTop{Detecting health misinformation in web page text}
  \TitleBottom{using deep learning methods}
  %\TitleBottom{Line 2 of the Tile} % edited Beeshanga
 \DegreeTitle{Bachelor of Engineering
 \\ Computer Engineering Stream} % edited Beeshanga

% Your research advisor
 \Advisor{Supervisor: Associate Professor Adam Dunn}

% The department undergraduate research coordinator
%  \UgradCoord{A}

% The representative of the department who will approve your thesis (usually the chair)
%  \DepRep{B}

% Acknowledge those who helped and supported you

  \Acknowledgments{
  \vspace{-1.5cm}
    \noindent I would like to acknowledge ...

  }


% The title of the department representative
%  \DepRepTitle{Chair}
  \Statement{
    \noindent I, (insert name here), declare that this report, submitted as part of the requirement for the award of Bachelor of Engineering in the School of Engineering, Macquarie University, is entirely my own work unless otherwise referenced or acknowledged. This document has not been submitted for qualification or assessment an any academic institution.
    \vspace{0.5cm}

    \noindent     Student's Name:

    \vspace{0.25cm}

    \noindent Student's Signature:

    \vspace{0.25cm}

    \noindent     Date:
    }

% The text of your abstract
\Abstract{
\vspace{-1.5 cm}
This is where you write your abstract ...

}



% Statement of Candidate



\fussy

\begin{document}

 % Start page counting in roman numerals
 \frontmatter

 % This command makes the formal preliminary pages.
 % You can comment it out during the drafting process if you want to save paper.

 \makepreliminarypages


%\clearemptydoublepage
\doublespace
%\include{Publications/publications}

% \clearemptydoublepage
%\include{Organization/organization}

 \clearemptydoublepage
\singlespace
 % Make the table of contents.
 \tableofcontents

\clearemptydoublepage
% Make the list of figures
\listoffigures

\clearemptydoublepage
% Make the list of tables
\listoftables

\clearemptydoublepage

% Start regular page counting at page 1
\mainmatter
%
\chapter{Introduction}
\label{chap:Introduction}

With the popularity and ubiquity of social platforms in today's society, the amount and the rate at which information is able to propagate online greatly outnumbers the manpower available that can evaluate the accuracy and determine the amount of misinformation within online articles. With factors such as the 'click-bait' nature and the lack of rigour surrounding the publishing of online content \cite{Sommariva2018} has caused an increase in the number of 'fake news' related content \cite{germanFN} \cite{Vosoughi}. This can be attributed to the trending or discover-based model commonly implemented by social media platforms that aim to maximize the reach and interaction of the content with no regards to the quality of the content's credibility. In specific domains, such as for health related articles, the spread of misinformation within a community can lead to the mistreatment and mismanagement of a range of health conditions which causes a wide variety of detrimental effects.  \newline

One of the key components required to minimize the propagation of misinformation online is to have the ability of automatically evaluating and quantifying the credibility of articles. However, traditional automated methods - such as shallow learning-based techniques, still require the domain knowledge of experts to be able to develop the features required by the model. Thus, this project aims to investigate the performance of Deep Learning-based (DL) techniques in evaluating the credibility of information within domain-specific articles via the classification of set criteria that have deemed to be highly correlated with articles that have low credibility. Specifically, this project will focus on evaluating the credibility of online health articles related to vaccination due to the commonly misinformed and controversial views associated with its effects \cite{Burgess2006}. \newline


\section{Project Overview}
This section details the scope of the project and its associated outcomes outlining the various tasks that must be accomplished to successfully complete the project.

\subsection{Project Scope}
\label{sec:ProjectScope}

The primary objective of this project is to evaluate the effectiveness of deep learning models in determining the credibility of online health-related articles. Due to the complexity of this project, a set of activities - divided into main goals and stretch goals, have been defined to ensure that the completion of this project remains feasible in the given time frame. The completion of all activities categorized as main goals will signal the realization of the primary objective and the completion of the project. The stretch goals are activities of interest that have been identified as non-essential to the completion of the primary objective but (talk about the overarching goal that all stretch goals have in common e.g. understand the model, utilize the model etc.) and will be worked on after the completion of the project.

\subsubsection{Main Goals}
\begin{itemize}
	\item Evaluate the performance of common ML-based methods for the classification of the 7 criteria in a specific domain
	\item Evaluate the performance of the chosen DL method for the same thing above
	\item Evaluate the effect of transfer learning methods in the performance of the DL method (assuming that the chosen method doesn't rely on transfer learning)
	\item Or maybe evaluate the performance of different transfer learning methods? e.g. zero vs few shot 
\end{itemize}



\subsubsection{Stretch Goals}
\begin{itemize}
	\item Utilize attention mechanisms to understand how the aforementioned DL model classifies the criteria for credibility.
\end{itemize}



\chapter{Background and Related Work}
\label{chap:LitReview}

\section{Credibility and Misinformation}
\textit{Talk about the work done in establishing the measurement of quality in online health information e.g. DISCERN, QIMR and that document from the slack channel}

\section{Prior Approaches}
\label{sec:PriorApproaches}

\textit{Discuss the prior work that has been done in terms of text classification e.g. spam, sentiment, topic}

\subsection{Shallow Learning Models}
\label{sec:MachineLearningModels}
\textit{Talk about commonly used ML models/shallow learning techniques for text classification e.g. SVM, Random Forest or Naive Bayes and justify which model(s) I will use as a baseline}

\subsubsection{Support Vector Machines}
\label{sec:SVM}

\subsubsection{Naive Bayes}
\label{sec:NaiveBayes}

\subsubsection{Artificial Neural Networks}
\label{sec:ANN}

\subsection{Feature Selection}
\label{sec:FeatureSelection}
\textit{Talk about word embeddings e.g. GloVe, word2vec, fastText, ngrams and its variants (skip-grams, sn-grams), BoW etc. and justify which features I will be using for this project.}

\subsubsection{Bag of Words}
\label{sec:BoW}

\subsubsection{N-Grams}
\label{sec:ngrams}

\subsubsection{GloVe}
\label{sec:GloVe}

\subsubsection{Word2Vec}
\label{sec:word2vec}

\subsubsection{Language Models}
\label{sec:LanguageModel}


\section{Deep Learning}
\label{sec:DeepLearningReview}
\textit{Introduce the state-of-the-art DL based approaches for text classification and try to compare it performance with state-of-the-art ML approaches} \newline

Deep learning models are a class of machine learning models that have the capability of automatically learning a hierarchical representation of data. These hierarchical representations are constructed through the use of artificial neural networks, the main underlying mechanism of deep learning models. Typically, large amounts of training data is required to train a model in learning the language model required to attain state of the art results, in the task of text classification for instance, the size of commonly used non-domain specific datasets range from hundreds of thousands of training examples to millions \cite{Conneau2017} \cite{Zhang} \textit{(note: look into the datasets used by state of the art approaches)}. Due to these constraints, it is not feasible to procure a dataset for the domain specific task of this project due to the aforementioned knowledge expertise and time requirements to manually label the articles required. Hence, \textit{(Talk about transfer learning/N-shot learning/domain adaptation here)} will be used to overcome this issue.

\textit{Introduce the typical architectures used for text classification e.g. RNNs, LSTMs, CNNs, GRUs?}

\subsection{Deep Learning Models}
\label{sec:DLModels}

\subsubsection{Recurrent Neural Networks}
\label{sec:RNN}

\subsubsection{Gated Recurrent Unit Networks}
\label{sec:GRU}

\subsubsection{Long Short-Term Memory Networks}
\label{sec:LSTM}

\subsubsection{Convolutional Neural Networks}
\label{sec:CNN}

\subsection{GET PROPER NAME FOR THIS SECTION}
\subsubsection{Transfer Learning}
\label{sec:TransferLearningReview}
\textit{Talk about transfer learning and how it works and how it is applicable to this project.}

\subsubsection{N-Shot Learning}
\label{sec:NShotLearningReview}
\textit{Talk about zero/few/etc-shot learning and how it works and how it is applicable to this project.}

\section{Conclusion}
\label{sec:LitReviewConclusion}
\textit{Summarize lit review and describe why DL-based approaches should be preferred over ML-based for this type of problem. Also talk about Transfer/N-Shot learning and describe which one will be feasible given the project's time constraints}

\chapter{Proposed Approach}
\label{chap:approach}

\section{Rationale}
\label{sec:ProposedRationale}
\textit{Introduce and discuss the factors that led to me choosing the proposed approach}


\section{Credibility Criteria}
\label{sect:CredibilityCriteria}
\textit{Introduce and discuss the 7 criteria that will be classified and describe how the criteria was determined}

\section{Study Data}
\label{sec:StudyData}
\textit{Talk about the data I'll be using, how we got it, its characteristics etc.}

\section{System Model}
\label{sect:chap2sysmodel}
\textit{Describe the architecture of the model}

\section{Experiments}
\label{sec:Experiments}
\textit{Describe the experiments that I'm planning to do (in such a way that they are easily reproducible)}

\section{Outcome Measures}
\label{sec:OutcomeMeasures}
\textit{Talk about the type of analyses that I'll be doing to determine the performance of my proposed model}

\chapter{Conclusions and Future Work}
\label{chap:Conclusions}


\section{Conclusions}
\label{sec:ConclusionsConclusions}

The end

\section{Future Work}
\label{FutureWork}



\clearemptydoublepage
\chapter{Abbreviations}
\label{chap:abbreviations}

\begin{tabbing}

AWGN \qquad \qquad \= Additive White Gaussian Noise\\
BC \> Broadcast Channel\\
BS \> Base Station\\
CSI \> Channel State Information\\
CSIR \> Channel State Information at Receiver\\
CSIT \> Channel State Information at Transmitter\\
dB \> Decibels\\
DPC \> Dirty Paper Coding\\
GS \> Gram-Schmidt\\
RVQ \> Random Vector Quantisation\\
SISO \> Single Input Single Output\\
SNR \> Signal to Noise Ratio\\
SINR \> Signal to Interference plus Noise Ratio\\
MISO \> Multiple Input Single Output\\
SIMO \> Single Input Multiple Output\\
MIMO \> Multiple Input Multiple Output\\
MMSE \> Minimum Mean Square Error\\
MRC \> Maximum Ratio Combining\\
QoS \> Quality of Service\\
TDD \> Time Division Duplex\\
FDD \> Frequency Division Duplex\\
ZF \> Zero-Forcing\\
ZFBF \> Zero-Forcing Beamforming\\
ZMCSCG \> Zero Mean Circularly Symmetric Complex Gaussian\\

\end{tabbing}

%\phantomsection \addcontentsline{toc}{chapter}{Index}
% \renewcommand{\baselinestretch}{1} \small \normalsize
% \printindex

\appendix
\chapter{name of appendix A}
\section{Overview}
here is the Overview of appendix A ...
\section{Name of this section}
here is the content of this section ...
\chapter{name of appendix B}
\section{Overview}
here is the Overview of appendix B ...
\section{Name of this section}
here is the content of this section ...

%\input{Bibliography/biblio3}
\bibliographystyle{IEEEtranS}
%\bibliographystyle{acm}
\bibliography{my_reference}
%\bibliography{Bibliography/biblio4}


\end{document}
